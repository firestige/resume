% !TEX TS-program = xelatex
% !TEX encoding = UTF-8 Unicode
% !Mode:: "TeX:UTF-8"

\documentclass{resume}
\usepackage{zh_CN-Adobefonts_external} % Simplified Chinese Support using external fonts (./fonts/zh_CN-Adobe/)
% \usepackage{NotoSansSC_external}
% \usepackage{NotoSerifCJKsc_external}
% \usepackage{zh_CN-Adobefonts_internal} % Simplified Chinese Support using system fonts
\usepackage{linespacing_fix} % disable extra space before next section
\usepackage{cite}

\begin{document}
\pagenumbering{gobble} % suppress displaying page number

\name{张宗泰}

\basicInfo{
  \email{zzt.nbl@hotmail.com} \textperiodcentered\ 
  \phone{(+86) 139-1860-2041} \textperiodcentered\ 
  \github[firestige]{https://github.com/firestige}}

\section{\faBriefcase\ 项目经历}

\datedsubsection{\textbf{GB28181协议联网网关·华为}}{2020年12月 -- 2024年6月}
\role{java, Linux}{MDE(模块设计)}
\begin{onehalfspacing}
作为组内MDE负责质量看护,核心功能设计和开发工作。组织代码评审,技术沙龙,配合主管完成技术梯队构建。
\begin{itemize}
  \item 主导完成常驻内存消减挑战,实现满规格下老年代占用缩减50\%,提升系统吞吐量20\%的目标,获得产品线十佳QCC改进荣誉。
  \item 利用Reactor模型改造消息收发模块,并且优化线程调度机制后,线程使用减少60\%,业务接口响应延迟降低40\%,获得产品线十佳微重构荣誉并在内部作为优秀案例推广。
  \item 配合主管完成模块质量爬坡工作,在组内推行开发者测试理念,完成单元测试废弃、迁移、新增用例约2000条。同时为测试团队开发压力测试工具,作为领域专家帮助测试人员设计压力测试场景9个。最终达成攻坚目标,获得亮剑奖荣誉。
\end{itemize}
\end{onehalfspacing}

\datedsubsection{\textbf{机数大材库·机数}}{2018 年3月 -- 2020年11月}
\role{Java, TS}{核心开发者}
\begin{onehalfspacing}
作为核心开发者负责数据挖掘模块与B/S应用设计和开发。
\begin{itemize}
  \item 利用朴素贝叶斯分类器、隐式马尔科夫模型分析全球化学专利文档,挖掘其中的化学反应方程式,为大材库凑齐初始的100万个有机化学反应方程。
  \item 基于Redis搭建化学命名同义词字典库,加速文档分析,缩短分析时间约20\%。并利用flink搭建微批处理流水线,实现化学文档抓取,分析加工,落库自动化流程。
\end{itemize}
\end{onehalfspacing}

\section{\faUsers\ 工作经历}
\datedsubsection{\textbf{华为} 杭州}{2022年5月 -- 至今}
\role{JAVA开发工程师}{职级:15}

\datedsubsection{\textbf{德科·华为OD} 杭州}{2020年12月 -- 2022年5月}
\role{JAVA开发工程师}{职级:D3}

\datedsubsection{\textbf{合肥机数量子科技有限公司} 合肥}{2018年2月 -- 2020年11月}
\role{JAVA开发工程师}{}

\datedsubsection{\textbf{南京普迪科技实业有限公司} 南京}{2013年3月 -- 2017年1月}
\role{工艺研发工程师}{}

\section{\faCogs\ IT 技能}
% increase linespacing [parsep=0.5ex]
\begin{itemize}[parsep=0.5ex]
  \item 语言运用:Java > python = TypeScript > Go
  \item 熟悉各类设计模式,需求分析方法,对spring、netty和Reactor源码有深入分析,有多个重构案例。
  \item 对CPU缓存机制和JVM有较深入了解,熟悉各种常用算法。
\end{itemize}

\section{\faGraduationCap\  教育背景}
\datedsubsection{\textbf{北京化工大学}, 北京}{2007 -- 2012}
\textit{学士}\ 应用化学

% \section{\faInfo\ 其他}
% % increase linespacing [parsep=0.5ex]
% \begin{itemize}[parsep=0.5ex]
%   \item 技术博客: http://blog.yours.me
%   \item GitHub: https://github.com/username
%   \item 语言: 英语 - 熟练(TOEFL xxx)
% \end{itemize}

\end{document}
